Upon opening a public account on RxChain, drug manufacturers will be
able to upload the name, descriptions, target population, prescription binary
requirement and clinical trial success of the drug,
effectively creating smart contracts embedded in the blockchain. These
standard contracts will have updatable parameters such as price,
quantity and repeated shipments, allowing for a tailored, sequential
delivery of drugs.
Once payment is received by the smart contract, drugs are dispatched
via a partner distribution company to the patient.
The contract will be fully executed upon payment receipt from the patient in RxCoins/Dollars
and the patient's confirmation of the drug's reception.
However we need to develop a more detailed protocol to ensure that the contract
validation is fault-tolerant. What if a malicious patient
receives the valid drug and refuses to validate the contract? We want
a system without external arbitration: enter complete factorial
graphs.
This new kind a smart contract imposes a fourfold dependency on three
parties so no party, or even coalition of parties will be able to
trick another.
DEPENDENCY GRAPH HERE:
The mechanism is dynamically summarised in the following graphs:\\
STEP I GRAPH HERE:\\
STEP II GRAPH HERE:\\
STEP III GRAPH HERE:\\

This protocol will instill trust and transparency into RxChains as well as streamline
manufacturer-distributor interactions.
Moreover, it  preserves the patients' anomnity from the viewpoint of manufacturers because
distributors will access to patients' sensitive information in a fashion similar
to how pharmacists access doctors' identification credentials during the prescription validation process.
Note that RxChain  will allow manufacturers to gain insight on real-time market demand for certain drugs,
as well as directly advertise, in real-time, through the RxChain price discounts. 


