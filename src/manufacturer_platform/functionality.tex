
Upon opening a public account on RxChain, drug manufacturers will be
able to upload the name, descriptions, target population, prescription boolean
requirement and clinical trial success of the drug,
effectively creating smart contracts embedded in the blockchain. These
standard contracts will have updatable parameters such as price,
quantity and repeated shipments, allowing for a tailored, sequential
delivery in time of drugs.
Once payment is received by the smart contract, drugs are dispatched
via a partner distribution company to the patient. Only after
receiving payment from a patient in RxCoins/dollars and the patient
validating adequate reception of the drugs will the contract be fully
executed.
However we need to develop a more detailed protocol to ensure that the contract
validation is fault-tolerant. What if the patient was malicious,
receives the valid drug and refuses to validate the contract? We want
a system without external arbitration: enter complete factorial
graphs.
This new kind a smart contract imposes a fourfold dependency on three
parties so no party, or even coalition of parties will be able to
trick another.
DEPENDENCY GRAPH HERE:
The mechanism is dynamically summarised in the following graphs:
STEP I GRAPH HERE:
STEP II GRAPH HERE:
STEP III GRAPH HERE:

This protocol will instill trust and transparency into RxChains as well as providing
simplicity to manufacturer-distributor interactions. Moreover, it  preserve the
patients anonyminity from the viewpoint of manufacturers as the
ditributors will have access to patients identity in a similar fashion
to how pharmacists could access doctors' identification credentials.
Notice that RxChain  will allow manufacturers to gain insight on real-time
market demand for certain drugs,
as well as directly advertise, real-time through the RxChain price discounts. 
Finally, manufacturers could purchase access to deeper client
analytics thanks to the multi-layered cryptographic capabilities of
patients,
as described in the following section.


