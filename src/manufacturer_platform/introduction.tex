Manufacturers do not control the flow of their product, passing
through the hands of various middlemen.
Because of this economic model, there is little competition between US pharma manufacturers.
In a 2014 report, the FDA criticizes the low productivity gains of US manufacturing plants: “Pharmaceutical manufacturing operations are inefficient and costly. Compared to other industrial sectors, the rate of introduction of modern engineering process design principles, new measurement and control technologies, and knowledge management systems is low.” 
Hence realizing direct manufacturer-to-patient transactions will unleash market forces in the US pharmaceutical industry, optimizing costs for all parties and driving innovation.
In particular manufacturers will gain control over its distribution,
information on its competitors and control of the end
market. Moreover, it will allow manufacturers to adjust, in real-time,
production of drugs according to demand as communicated by the
RxChain, rather than leaving drug surpluses/shortages accumulate in
the supply chain.
Overall, it will dramatically increase manufacturers’ power as
economic actors in the pharmaceutical drug market and increase their profitability.
