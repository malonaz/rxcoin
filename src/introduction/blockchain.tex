Blockchain is a decentralized digital ledger in which data can be written and recorded chronologically and publicly.
In other words, blockchain is a protocol for a distributed database, continuously reconciled by millions of miners.
Miners are the computers that power this decentralized database.
When a user wishes to write data, he uses the blockchain's proprietary protocol to broadcast his action on the database
to the network of miners.
There, the miners pool many of these operations (e.g. transactions or contracts) into a computer data structure called a block,
and then compute a pre-defined (by the blockchain) arbitrary problem with this block as input,
that is known to be computationally expensive.
Discussion beyond basic concept of these computational so-called 'proof-of-work' is not necessary at this point.
The first miner to find a a solution to the aforementioned arbitrary problem can publish it to the network.
This block is now added to the blockchain and the miner received a pre-fixed sum of bitcoins as a reward.
