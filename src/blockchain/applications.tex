Earlier, we repeatedly mentioned \emph{operations} rather than \emph{transactions} intentionally.
When speaking about the infamous Bitcoin, it is accurate enough to speak of \emph{transactions} as its blockchain
only allows a small set of transaction-related operations.
Bitcoin and its blockchain thus form a cryptocurrency, with limited use beyond transacting at poor lentency.

In comes Ethereum...
Whereas Bitcoin restricts it blockchain to transaction-related operations, Ethereum has embraced all operations.
Ethereum designed a coding language, made up of primitives called \emph{instructions} that Ethereum mining software can interpret.
An instruction is a primitive such as ADD, STORE, LOAD, JUMP. (This will be important later)
Their set of instructions is turing-complete.
A system is said to be turing complete if it can be used to solve any computable problem.

Ethereum's blockchain is thus a platform rather than a cryptocurrency.
A user is free broadcast any data and/or code to the platform, and as miners validate a block,
they execute this code to update the state of the database.
These codes, often referred to as \emph{smart contracts}, can call on other contracts, or be called upon.

Since miners must compute these smart contracts'code, what happens if a user decides to spam the mining network with
smart contracts that contain useless code loops. 
Any miner that picks such an operation from the pool of pending operations will not be able to process the block.
Such an attack is known as a \emph{distributed denial of service}.
In order to prevent such an attack, Ethereum charges a fee per code instruction, paid in their digital currency called Ether.
Ether is used to pay miners and pay for transactions.
Ethereum's blockchain keeps the balances of users in its decentralized database.
Hence, if a user sends a smart contract from an address, he must have enough Ether at this address to see his contract's code
to completion.
If a address that initiates a code runs out of Ether before the code is complete, the database is rolleed back to its prior
state, except the code fees which is still pocketed by the miner for his time.

Using this framework, anyone can create a application that uses Ethereum's decentralized database.
Today, most of the cryptocurrencies or ICOs you read about, have not implemented a blockchain themselves,
but rather, use Ethereum's blockchain.

