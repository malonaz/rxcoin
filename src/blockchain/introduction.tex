Blockchain is a decentralized digital ledger in which data can be written and recorded chronologically and publicly.
In other words, blockchain is a protocol for a distributed database, continuously reconciled by its mining community.
Miners are the computers that power this decentralized database.
When a user wishes to write data, he uses the blockchain's proprietary protocols to broadcast an operation on the database
to the network of miners.
There, the miners pool many of these operations (e.g. transactions or contracts) into a computer data structure called a block.
They then race to compute a pre-defined (by the blockchain) arbitrary problem with this block as input,
that is known to be computationally expensive.
Discussion beyond basic concept of these computational so-called 'proof-of-work' will be covered later.
The first miner to find a a solution to the aforementioned arbitrary problem can publish it to the network.
This block is now added to the blockchain, and race to the next blocks continuees.
Upon publishing a block, the miner receives a pre-determined sum of the blockchain's coin as a reward.
It is this reward that incentivizes miners to power the network.
