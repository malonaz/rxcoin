In this section, miners will also be referred to as nodes, the graph all nodes is called the RxCoin network.At any given point,each node has two sets of information: a ledger of blocks on which they have reached a consensus and an outstanding "pool" of blocks that have not yet been validated by the network. We will assume 
Say Alice wants to perform a transaction.We assume Alice honest. Then Alice broadcasts the transaction to whole RxCoin network. Some nodes "hear" about Alice's broadcast and her transaction enters the oustanding "pool" of these nodes.
Then, every $x$ minutes, a node is randomly selected and broadcasts a block containing Alice's transaction. The other nodes accept the block only if it is valid. They accept the block by including its hash in the next block they create. If the randomly selected node is malicious,its block is not built upon and another randomly selected node's block will be favoured. The blockchain consists of the longest chain i.e the most accepted version of all transactions. \\

In this scenario, we have made two assumptions: that Alice is honest and that the majority of nodes are honest. 
Let's test these assumptions. If Alice was not honest,and say tried to steal RxCoins then if a node includes her transaction in a block this is equivalent to the above scenario with a malicious node. \\


Now let's turn to the problem of the majority of nodes being honest, also named the Byzantine Fault-Tolerance problem.
To illustrate this problem, consider if 51-percent of nodes were malicious. Then, because the decentralised protocol consensus relies on selecting randomly proposed options until a majority agrees (remember longest chain), a malicious majority could allow invalid transactions to happen.
Hence there are two systems to incentivise nodes to be honest: rewards and punishments.\\

Rewards come in two forms: RxCoin tokens and 
The higher the dollar exchange rate, the greater the incentive.
However, if we distribute too generously RxCoin, we risk devaluing it by conducting an inflationary policy.
Hence we must limit the supply of RxCoins and typically after a number of initial coins in circulation, the number of new coins issued should decrease pseudo-exponentially each year. We also have to limit the demand for mined RxCoins by introducing capital hurdles to the mining activity, this is through mathematical puzzles: this is called proof-of-work (PoW).\\

PoW must have problems that are hard to solve but with easily validated  solutions– in other words PoW relies on NP-hard computational puzzles.
The way PoW works is the following: the random selection of nodes is actually skewed in favour of those with greater computing power. Indeed, the first node to have solved the puzzle is allowed to broadcast their block, giving weight to technological advantage. At a certain level of difficulty , puzzles will be solved as fast on a PC than on specialised hardware. However, if we increase the level of difficulty as the number of miners increase and the supply of RxCoin diminishes, the upfront capital costs required by hardware upgrades will render small miners uncompetitive. This risks driving a centralisation of mining and thus jeopardise the decentralised nature of RxCoin's network, making it vulnerable to 51-percent attacks– i.e majority controlled by a malicious nodes.\\

To alleviate this, we can implement "punishments" for malicious nodes or modify the type of puzzle so its solving is less specialized-hardware dependent. We will not go at lenght into these solutions and instead discuss another mining architecture: Proof of Stake.\\
