Let's begin with some background knowledge.\\
Blockchain is a decentralized digital ledger in which data can be written and recorded chronologically and publicly.
In other words, blockchain is a protocol for a distributed database, continuously reconciled by millions of miners.
Miners are the computers that power this decentralized database.
When a user wishes to write data, he uses the blockchain's proprietary protocol to broadcast his action on the database
to the network of miners.
There, the miners pool many of these actions (e.g. transactions or contracts) into a computer data structure called a block,
and then compute a pre-defined (by the blockchain) arbitrary problem with this block as input,
that is known to be computationally expensive.
Discussion beyond basic concept of these computational so-called 'proof-of-work' is not necessary at this point.
The first miner to find a a solution to the aforementioned arbitrary problem can publish it to the network.
This block is now added to the blockchain and the miner received a pre-fixed sum of bitcoins as a reward.

How are addresses generated? 
Any user may generate any number of private keys free of cost.
A series of cryptographic hash functions are applied to the private key to derive a public key from it.
To put it bluntly, hash functions are functions f(x) = y, such that given x, it is computationally inexpensive to find y,
but infeasible to compute x given y.
This public key is hashed for compression purposes to form an address.
This address is what will be present in the decentralized database, for any of your operations.


\begin{center}
\includegraphics[scale = 0.25]{diagrams/transaction_diagram.png}
\end{center}


Let us now look at how a transaction works.



Let us now briefly discuss a few important applications of blockchain.
The infamous Bitcoin blockchain is primitive compared to new applications.
Its protocol only allows users of its database to make certain operations such as paying or 


RxChain is the name of the blockchain technology
